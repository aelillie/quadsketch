\section{Analysis}
\label{analysis}
%Our analysis of the paper and algorithm goes here
%From the course's intended learning outcomes (ILO):
%"Find, extract and explain results in the algorithms research literature relevant to a given problem"
%"Theoretically analyze the performance of a given algorithmic solution"
This section provides an analysis of properties of the \qs{} algorithm as introduced in \cite{wagner17}, which are deemed interesting and relevant for understanding the algorithm, and thus extend it with an improvement.

%TODO: Do we have more analysis?

\subsection{Why randomly shift hypercube?}
Randomly shifting a hypercube in each dimension is necessary for the \qs{} implementation to maintain guarantees on arbitrary datasets. A good result for the cube would be a cube in which data points near each other are shifted into the same leaves in the \qt{}. The shifting can have varying practical effects given the randomness of the of the approach.
\\
\\
If the dataset is of a known format, randomly shifting it in each dimension could turn out to be less effective than taking a more specific approach. An example could be a dataset in which all the points are very close to each other. A large amount of the points might end up in the same leaves and as such defeat the purpose of using the \qt{} to begin with. Randomly shifting over these points will not be of any significant gain to the problem at hand.

\subsubsection{Aspect ratio}
\label{sec:aspect_ratio}
The aspect ratio is defined as the ratio between the largest and smallest distance between points in the dataset\cite{wagner17}. The aspect ratio is denoted as $\Phi$:
\\
\\
$$ \Phi = \frac{max_{1<i<j<n} || x_i - x_j || }{min_{1<i<j<n} || x_i - x_j || } $$
\\
\\
The aspect ratio can be used to gain insight about the distribution of data points in euclidean space. When $\Phi$ approaches 1 then points are clustered close together. When $\Phi$ grows then the points are more spread out in the point space but will not describe how the points are spread. There could be an outlier which gives a big $\Phi$. If some point set p is distributed in a euclidean space, the quad sketch will need more bits to precisely describe the position of points in p. Hence $\Phi$ has an impact on how the parameters $\Lambda$ and $\L$ should be set to dictate the guarantees of the sketch.
\\
\\
When $\Phi$ grows, so does the required value for both $\Lambda$ and $\L$ if the guarantees given in the theorems are to hold. A large aspect ratio describes a point set in which at least one point is far from the other points relative to the distance between the other points. This results in a need to store more information in order to retain the relative distances between the points in the sketch.

\subsection{Theorems}
The paper introduces three theorems, where \tm{1} is the main theorem covering the basic variant of \qs{}. \tm{2} covers a variant which focuses on minimizing the distortion between all points, and \tm{3} introduces a block variant of \tm{1} inspired by \cite{schmid9}.
\\
\\
Due to some ambiguity between \tm{1} and \tm{2}, it is assumed that for \tm{1} the sketch is created from a single point, and from that point the distances to all other points are preserved up to a factor of 1 $\pm$ $\epsilon$. In \tm{2} it is mentioned that the algorithm is applied recursively in order to preserve the distances between \textit{all} pairs with a high probability.
\\
\\
This ambiguity arises due to several factors. Firstly it is stated in \tm{1} that \textit{foreach point} all distances from that point to all other points are preserved, which would entail that for any point it is possible to retrieve the approximate distance from that point to all others. In \tm{2} it is since highlighted that this theorem makes it possible to approximate the distance between \textit{all pairs of points}. This similarity between the two theorems makes it difficult to exactly analyze the differences between them. Secondly there is no probability guarentee provided in \tm{2}. Without this bound it becomes cumbersome to deduct the cost and the actual value of the recursive process introduced in \tm{2}. Lastly does it not seem logical to introduce \tm{2} as it is not used later in the following theorem, which only builds on top of \tm{1}, nor is it not used throughout the remainder of the paper. 
\\
\\
It could be assumed that \tm{1} creates a sketch where it is possible to preserve the distance between all points similarly to \tm{2}, all though this does not seem likely. The reasons described above argues against it as well as the following property: it is possible to decompress the sketch produced by \tm{1} back into in approximate pointset, whereas this is not possible for \tm{2}. Hence should \tm{1} and \tm{2} both allow for all points to preserve the approximate distances to all other points, it becomes uncertain exactly what the differences between these theorems are. 
%TODO: Properly review theorem section