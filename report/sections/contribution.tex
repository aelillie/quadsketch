\section{Contribution}
\label{contribution}
%Short section about our final findings and contributions
%This section describes the contributions given to \qs{} as the results for this project.
This section will briefly present the results of the project\footnote{See \url{https://github.com/aelillie/quadsketch} for the GitHub repository for the project}. Through experiments on SIFT and MNIST with the \qs{} implementation, as well as a version of \gr{} it is observed that the results are comparable to those reported in \cite{wagner17}. 

Furthermore an extension of \qs{} called \qsr{} (for QuadSketch Random, in lack of better names) is implemented, where the decompression step has been modified. Instead of placing zeros in the bit representation of a point, which is deemed "non-important" to the compression scheme(called \textit{long edges} in the paper), bits are inserted at random. Empirical experiments show slight improvements from \qsr{} over \qs{}. The most significant of which is for the experiments on the \mnist{} dataset, where \qsr{} overall performs better than \qs{}. The average result reached for each block size in the tests are summarized in tables \ref{table:avg_mnist_qs1} and \ref{table:avg_mnist_qsr1}, where high accuracy and low distortion with minimum size is better. Note however that the implemented modification does not alter the sketch size, which is influenced by other factors. Even though the algorithm in its core is highly influences by some randomness(see section \ref{qs}), the results gained from the experiments are quite convincing. In section \ref{results}, this is further demonstrated in figure \ref{fig:graph mnist}, which visualizes the test results in a graph. Additionally an expected performance gain is obtained by the fact that random bit insertion ensures a lower distortion of pruned points, it is presented in section \ref{exp:dist}.

%TODO: Insert probability or distance improvements with QSR

\begin{table}[h]
	\centering
	\caption{Average Results for \qs{} on \mnist{}}
	\label{table:avg_mnist_qs1}
	\begin{tabular}{l l l l}
		\hline
		\#Blocks & Bits per coordinate & Accuracy  & Distortion \\ \hline
		2 & 3.43 & 0.44 & 1.100  \\
		4 & 4.36 & 0.51 & 1.096  \\
		7 & 3.75 & 0.53 & 1.097 \\
		8 & 3.69 & 0.56 & 1.09 \\
		14 & 3.51 & 0.67 & 1.048 \\
		16 & 3.45 & 0.69 & 1.039 \\
		28 & 3.35 & 0.81 & 1.013 \\
		\hline
	\end{tabular}
\end{table}

\begin{table}[h]
	\centering
	\caption{Average Results for \qsr{} on  \mnist{}}
	\label{table:avg_mnist_qsr1}
	\begin{tabular}{l l l l}
		\hline
		\#Blocks & Bits per coordinate & Accuracy  & Distortion \\ \hline
		2 & 3.79 & 0.55 & 1.118  \\
		4 & 3.73 & 0.57 & 1.09  \\
		7 & 3.21 & 0.62 & 1.062 \\
		8 & 3.16 & 0.64 & 1.052 \\
		14 & 3.01 & 0.74 & 1.023 \\
		16 & 2.95 & 0.76 & 1.02 \\
		28 & 2.87 & 0.85 & 1.008 \\
		\hline
	\end{tabular}
\end{table}
