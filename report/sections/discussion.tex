\section{Discussion}
\label{discussion}
This section discusses the findings presented in section \ref{results} and evaluates how the results relate to the goals stated in section \ref{intro:problemStatement}. Furthermore possible threats to validity of the results will be discussed.

\subsection{Verification of Performance}
The results in figures \ref{fig:graph sift} and \ref{fig:graph mnist} are used for verifying the results presented in \cite{wagner17}, whose results shows that \qs{} outperforms the baseline \grid{} algorithm for the datasets \sift{} and \mnist{} in regards to both accuracy and distortion. In both figures it is clear that \qs{} indeed performs better than \grid{} and therefore confirms this trend. As the precise results in \cite{wagner17} have not been published, no distinct quantitative measure of the "similarity" between the graphs have been used. As such it is not possible to construct a similarity measure between any two nodes from the original graphs and the graphs created in this paper. 

Apart from these obstacles, it is clear that the graphs are very similar, thus indicating that the results presented in \cite{wagner17} are accurate. It should be mentioned for clarity that outliers on the figures are the result of running \qs{} with inappropriate parameters e.g. there is pruned too much information. This process follows the original in \cite{wagner17}. These outliers help verify that the accuracy and distortion of the sketch is indeed controlled by the parameters \textit{L} and $\Lambda$.  

\subsection{Quadsketch and Quadsketch Random}
This subsection will focus on the comparison between \qs{} and \qsr{} as some interesting properties appear from the results. 

Firstly both algorithms obtain approximately the same best scores for the datasets, except \mnist{} where \qsr{} achieves better results. Secondly \qsr{} displays a better average case than \qs{} empirically (figures \ref{fig:graph sift}, \ref{fig:graph mnist} and \ref{fig:graph clust}), which aligns with the expectation from the theoretical improvements behind \qsr{}. The improvements are covered in section \ref{exp:dist}, where the random bit insertion in \qsr{} entails a lower expected distortion of the original point set in regards to pruned points.

The reason \qsr{} outperforms \qs{} on \mnist{} is due to the large amount of dimensions in the \mnist{} dataset. From datasets with higher dimensionality it follows that longer bit strings will be pruned and replaced by \qs{}. Decompressing longer bit strings gives the random bit replacement a higher chance of converging towards the expected value. The lower expected value for \qsr{} bit replacement compared to \qs{} gives \qsr{} a better chance to obtain better accuracy and lower distortion. Even though the final result is data dependent, \qsr{} is likely to bring improvements over \qs{} on arbitrary highly dimensional data or cases where a lot of pruning will occur.

\subsection{Threats to validity}

\subsubsection{Randomness}
\label{disc/threats/randomness}
The comparison of the algorithms shown in section \ref{results} show single test runs of the algorithms with different sets of parameters. The seemingly positive results for \qsr{} are inherently subdued to some randomness and could vary on further test runs. The same also applies to \qs{} as it makes use of a randomly shifted grid, which can have effect on the outcome of single runs as well.

\subsubsection{Depth of quad in experiments}
\label{disc/threats/depth}
In \cite{wagner17} the experiments on \qs{} and \grid{} runs with the parameter $\Lambda$ 1 to 19 and \textit{L} 2 to 20. This project stops the $\Lambda$ at 9 and \textit{L} at 10 because the running time of \qs{} with larger parameters has not been feasible for this project. As the results of this paper already discloses both accuracy and distortion values very close to one, the expected improvements gained by increasing the parameters beyond the current bound are very limited. However it is possible that increasing the parameters would yield results that could reveal more information or other differences in the algorithms.

\subsubsection{Missing Dimensionality Reduction}
There is a slight offset between the results in \cite{wagner17} and the results in this paper. \cite{wagner17} uses the Johnson-Lindenstrauss lemma to reduce the number of dimensions in a dataset to be $d=$\bo{$\epsilon^2\log(n)$}. A slight concern arises because it is unclear if the datasets \mnist{} and \sift{} used in this paper has been preprocessed with Johnson-Lindenstrauss. The concern derives from the \qs{} implementation provided on GitHub, because this implementation does not apply Johnson-Lindenstrauss as part of the source code. Therefore some investigation has been done in this area, and although not confirmed it seems that the datasets have not undergone any preprocessing. A supportive argument entails that it would be natural to separate these two processes, as this yields a more clean implementation of \qs{}. 
\\
In the given case that Johnson-Lindenstrauss has not been applied to the datasets, the offset is natural (and should be expected), as the number of dimensions in a dataset has a direct impact on the number of bits required to represent a point. If it is not the case, then the results obtained in this paper differs from the results obtained in \cite{wagner17}.