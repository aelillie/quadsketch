\section{Future Work}
\label{futurework}
This section will discuss what future work can be done as an extension to this paper and to \cite{wagner17}. Two new contributions will be shortly introduced and their main obstacles described. 

\subsection{Dynamic Addition}
Through the creation of this project some discussions has taken place regarding the possibilities of dynamically adding new points to a finished sketch. The algorithm introduced in \cite{wagner17} only considers computing a sketch from a static dataset and adding a new point to an existing sketch would therefore require a complete recomputation. Constructing the sketch is a very expensive operation, especially for datasets with many dimensions and a high aspect ratio, as these require a larger tree to obtain appropriate metrics. This high construction time limits the application of \qs{} to static datasets. The ability to dynamically add points would allow for further usages of \qs{}, and would thus be a valuable contribution.  

\subsection{Automatic Parameters}
The implementation provided by Tal Wagner requires several user provided parameters. For users with limited knowledge about the mechanics of the algorithm, this poses a problematic entrance. However it is possible to reduce these parameters. The user could be asked to simply provide the desired distortion guarentee. Given this parameter, the algorithm could calculate the required depth \textit{L} and pruning parameter $\Lambda$ to meet the guarentee by first calculating the aspect ratio $\Phi$ of the dataset. Calculating $\Phi$ is a simple but very expensive operation as it requires finding the largest distance between two points in the given dataset. This would extend the current running time of \qs{} with \bo{n$^2$} where \textit{n} is the number of points in the dataset. 