\section{Introduction}
\label{introduction}
%Formal introduction goes here
Compression algorithms for approximate nearest neighbor search is a well studied field and many papers have been published on the area. We will study one paper in particular, where the key idea is to make use of a multi-dimensional indexing method known as hyperoctrees \footnote{These are referred to as \qt{}'s in the paper} and apply pruning to reduce the size of the resulting sketch. The result of the paper is an algorithm known named \qs{}, which represents each point in a Euclidaen space using only \bo{d*log(dB/$\epsilon$) + log(n)}. Alongside with the paper the authors released an implementation of the algorithm, which is publicly available on Github\footnote{https://www.Github.com}.
\\
\\
The paper introduces several experiments demonstrating the efficiency of \qs{}. The authors have tested \qs{} along side with two other approximate nearest neighbor algorithms, \texttt{Product Quantization} referred to as \pq{} and a scalar uniform quantization implementation referred to as \gr{}. 
\subsection{Motivation} %Why are we interested in this?

\subsection{Case} %What is the QuadSketch paper about?

\subsection{Problem definition} %What is our research question?
%From ILO:
%"Plan and carry out a small-scale investigation of an algorithmic research problem. This investigation could be theoretical, experimental, or both."
