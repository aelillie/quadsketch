\section{Introduction}
\label{introduction}
%Formal introduction goes here
\subsection{Motivation} %Why are we interested in this?
Compression algorithms for approximate nearest neighbor is a well studied field and many papers have been published on the area. The need for compression arises as many nearest neighbor algorithm suffer from \textit{"the curse of dimensionality"} phenomenon; either space or query time are exponential in the dimension \textit{d}. To escape this curse,
researchers proposed approximation algorithms for the problem.\cite{ilya15} 
\\
\\
Approximation algorithms often work on compact distance-preserving representations, and these are usually divided into two categories: data-oblivious and data-dependent. The former attempts to achieve guarentees for any data set while the latter attempts to use the extra information about the particular data sets in order to design functions with better performance. Compressed representations makes computation for data analysis algorithms more efficient, which is very desirable in many fields.
\subsection{Case} %What is the QuadSketch paper about?
We will study one paper in particular, where the key idea is to make use of a multi-dimensional indexing method known as hyperoctrees\footnote{These are referred to as \qt{}'s in the paper} and apply pruning to reduce the size of the resulting sketch. The result of the paper is an algorithm known named \qs{}, which represents each point in a Euclidaen space using only \bo{d*log(dB/$\epsilon$) + log(n)} where \textit{d} is the result of reduing the number of dimensions \textit{D} of a pointset by \bo{$\epsilon^{-2}$ log n} and \textit{B} is the number of bits of precision. Alongside with the paper the authors released an implementation of the algorithm, which is publicly available on Github\footnote{https://www.Github.com}. 
\\
\\
The paper introduces several experiments demonstrating the efficiency of \qs{}. The authors have tested \qs{} along side with two other approximate nearest neighbor algorithms, \texttt{Product Quantization} referred to as \pq{} and a scalar uniform quantization implementation referred to as \gr{}.
\subsection{Problem definition} %What is our research question?
%From ILO:
%"Plan and carry out a small-scale investigation of an algorithmic research problem. This investigation could be theoretical, experimental, or both."
